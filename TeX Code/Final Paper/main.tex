% !TEX program = xelatex
\documentclass[UTF8]{article}


% 导入宏包
    % 版式
    \usepackage[a4paper]{geometry}    % 设定版式为 a4
    
    % 中文
    \usepackage{ctex}

    % 绘图相关
    \usepackage{tikz}    % 绘图宏包
    \usepackage[enableskew,vcentermath]{youngtab}    % 杨表绘图,范例见 young tableau 部分
    \Ylinethick{1.5pt}    % 杨表的线段粗细
    \Yautoscale{0}
    \Yboxdim{14pt}

    % 表格
    \usepackage{longtable,booktabs,array}    % 三线表和长表格
    % \usepackage{tabularx}
    % \usepackage{xltabular}    % 跨页长表格+换行,有需要请去掉注释

    % 图片
    \usepackage{graphicx}
    \usepackage{float}    % 浮动体控制

    % 链接
    \usepackage{url}    % 网页链接


% 文档基本信息
    % 导言区
    \title{课程论文}
    % 注意字母序
    \author{张三(0000000) \ 李四(1111111)}
    \date{\today}




\begin{document}

% 表格页
\begin{center}
    \section*{“离散结构及其算法”课程论文说明表}
    \begin{tabular}{|c|c|}
        \hline
        作者信息(学号-姓名)  & ~ \\ \hline
        论文题目         & ~ \\ \hline
        论文内容确定的依据和想法 & ~ \\    \hline
        论文的创新点       & ~ \\ \hline
        备注           & ~ \\ \hline
    \end{tabular}
\end{center}
\clearpage

\maketitle

\begin{abstract}
    摘要内容

    \textbf{关键词}: 离散结构及其算法
\end{abstract}

\section{正文开始}
正文结束。


% 简单的参考文献表,也可自行使用 BibTeX
\begin{thebibliography}{99}

    % 以下为 参考文献 的一个示例
    \bibitem{ref1}郭莉莉,白国君,尹泽成,魏惠芳. “互联网+”背景下沈阳智慧交通系统发展对策建议[A]. 中共沈阳市委、沈阳市人民政府.第十七届沈阳科学学术年会论文集[C].中共沈阳市委、沈阳市人民政府:沈阳市科学技术协会,2020:4.
\end{thebibliography}

\end{document}
