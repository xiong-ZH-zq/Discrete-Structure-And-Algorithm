% !TEX program = xelatex
\documentclass[UTF8]{article}


% 导入宏包
    % 版式
    \usepackage[a4paper]{geometry}    % 设定版式为 a4
    
    % 中文
    \usepackage{ctex}

    % 绘图相关
    \usepackage{tikz}    % 绘图宏包
    \usepackage[enableskew,vcentermath]{youngtab}    % 杨表绘图,范例见 young tableau 部分
    \Ylinethick{1.5pt}    % 杨表的线段粗细
    \Yautoscale{0}
    \Yboxdim{14pt}

    % 表格
    \usepackage{longtable,booktabs,array}    % 三线表和长表格
    \usepackage{tabularx}
    % \usepackage{xltabular}    % 跨页长表格+换行,有需要请去掉注释

    % 图片
    \usepackage{graphicx}
    \usepackage{float}    % 浮动体控制

    % 链接
    \usepackage{url}    % 网页链接

    % 附录
    \usepackage{appendix}

    % 代码块
    \usepackage{listings}
    \lstset{
        basicstyle          =   \sffamily,          % 基本代码风格
        keywordstyle        =   \bfseries,          % 关键字风格
        commentstyle        =   \rmfamily\itshape,  % 注释的风格,斜体
        stringstyle         =   \ttfamily,  % 字符串风格
        flexiblecolumns,                % 别问为什么,加上这个
        numbers             =   left,   % 行号的位置在左边
        showspaces          =   false,  % 是否显示空格,显示了有点乱,所以不现实了
        numberstyle         =   \zihao{-5}\ttfamily,    % 行号的样式,小五号,tt等宽字体
        showstringspaces    =   false,
        captionpos          =   t,      % 这段代码的名字所呈现的位置,t指的是top上面
        frame               =   lrtb,   % 显示边框
    }

    \lstdefinestyle{Python}{
        language        =   Python, % 语言选Python
        basicstyle      =   \zihao{-5}\ttfamily,
        numberstyle     =   \zihao{-5}\ttfamily,
        keywordstyle    =   \color{blue},
        keywordstyle    =   [2] \color{teal},
        stringstyle     =   \color{magenta},
        commentstyle    =   \color{green!30!black!70}\ttfamily,
        breaklines      =   true,   % 自动换行,建议不要写太长的行
        columns         =   fixed,  % 如果不加这一句,字间距就不固定,很丑,必须加
        basewidth       =   0.5em,
    }


% 文档基本信息
    % 导言区
    \title{课程论文}
    % 注意字母序
    \author{张三(0000000) \ 李四(1111111)}
    \date{\today}




\begin{document}

% 表格页
\begin{center}
    \section*{“离散结构及其算法”课程论文说明表}
    \begin{tabular}{|c|c|}
        \hline
        作者信息(学号-姓名)  & ~ \\ \hline
        论文题目         & ~ \\ \hline
        论文内容确定的依据和想法 & ~ \\    \hline
        论文的创新点       & ~ \\ \hline
        备注           & ~ \\ \hline
    \end{tabular}
\end{center}
\clearpage

\maketitle

\begin{abstract}
    摘要内容

    \textbf{关键词}: 离散结构及其算法
\end{abstract}

\section{正文开始}
正文结束。
杨表绘图:

\begin{center}
    \(
    \young(~:~:2:~,~:~:~:,~:\bullet:~:,~:~::,~:::)
    \)
\end{center}



% 简单的参考文献表,也可自行使用 BibTeX
\begin{thebibliography}{99}

    % 以下为 参考文献 的一个示例
    \bibitem{ref1}郭莉莉,白国君,尹泽成,魏惠芳. “互联网+”背景下沈阳智慧交通系统发展对策建议[A]. 中共沈阳市委、沈阳市人民政府.第十七届沈阳科学学术年会论文集[C].中共沈阳市委、沈阳市人民政府:沈阳市科学技术协会,2020:4.
\end{thebibliography}


\clearpage
\begin{appendices}
    % 附录页
    \section*{附录}
    \subsection*{本文算法代码实现}

    \lstinputlisting[
        style       =   Python,
        caption     =   {\bf demo.py},
        label       =   {demo.py}
    ]{./src/demo.py}



\end{appendices}



\end{document}
